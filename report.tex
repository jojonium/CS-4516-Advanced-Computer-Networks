\documentclass[a4paper, 11pt]{article} % 11 pt font
\usepackage[utf8]{inputenc}
\usepackage[american]{babel}
\usepackage[margin=1in]{geometry} % 1 in margins
\usepackage{csquotes}
\usepackage{lipsum}

\usepackage{fancyhdr}
\setlength{\headheight}{15.2pt}
\pagestyle{fancy}
\lhead[]{Cole Granof and Joseph Petitti}
\rhead[]{Advanced Computer Networks}
\chead[]{Project Report}

\usepackage[
	backend=biber,
	style=ieee
]{biblatex}

\bibliography{references}

\title{Advanced Computer Networks Project Report}
\author{Cole Granof and Joseph Petitti}
\date{\today}

\begin{document}
\thispagestyle{empty}

\maketitle

\section{Overview}

\textbf{What the project consisted of and what you implemented.} This project
was inspired by a paper we read for class \cite{tscm16}.

\lipsum


\section{Phase Details}

This section describes the implementation of each phase of the project.

\subsection{Phase 1}

How was the gateway configured?

\lipsum

\subsection{Phase 2}

How was traffic logging implemented?

\subsection{Phase 3}

\subsubsection{Classification Vectors}

While we were in the process of trying to make the machine learning component as
accurate as possible, we continually added features in the hopes of improving the
accuracy of the model. Our feature vector originally contained only five elements,
and expanded to fourteen elements as we iterated on our strategy for machine learning.

The following elements within the feature vector are as follows:

\begin{itemize}
    \item Byte ratio (bytes sent / bytes received, or reciprocal)
    \item Packet ratio (packets sent / packets received, or reciprocal)
    \item Mean packet length
    \item Standard deviation of packet lengths (zero if n <= 2)
    \item Packet length kurtosis
    \item Packet length skew
    \item Mean time gap between each packet (zero if n <= 1)
    \item Time gap kurtosis
    \item Time gap skew
    \item Min packet length
    \item Max packet length
    \item Min time gap
    \item Max time gap
    \item Protocol (1 for TCP, 0 for UDP)
\end{itemize}

As you can see from the list above, the bulk of the features are statistical measure
for two main lists.

We collected four rounds of packet traces for each of the applications being tested.
Each round of training data collection took around thirty minutes to an hour. Certain
applications generated many megabytes of data. This was especially true for applications
that regularly streamed video, such as YouTube. Other applications generated far less
traffic, such as Fruit Ninja, which played ads at relatively infrequent intervals.


(What classification algorithm did you use? Which features? How many traces did
you use for training and evaluation? Which accuracy did you obtain?)

\subsection{Phase 4}

How was the classifier integrated into the traffic logger?

\lipsum

\section{Lessons Learned and Limitations}

Describe anything that you tried during any of the phases that did not work.
Also describe limitations of your implementation and anything that did not
perform according to your expectations.

\newpage

\printbibliography

\end{document}
